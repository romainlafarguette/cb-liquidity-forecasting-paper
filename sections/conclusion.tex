\section{Introduction}
Central banks forecast liquidity to  implement monetary policy. “Liquidity” is
defined as on-call  account of monetary counterparties (banks)  at the central
bank,  also call  “reserves.” Liquidity  forecasts are  an important  input to
calibrate monetary  operations, the  objective of which  is to  align monetary
conditions with  the announced  monetary policy stance.  This applies  to most
monetary policy frameworks and is  particularly important when the operational
target is short-term interest rates.

Statistical  liquidity forecast  consists in  modelling the  behaviors of  the
central  bank  non-monetary  counterparties.   Those  counterparties  are  not
supervised by the central  bank and do not have access  to the monetary policy
operations of the central bank. Central banks can collect information directly
from  these  counterparties. This  is  the  non-statistical component  of  the
liquidity forecast also called “institutional arrangements.”  However, central
banks  also  use  time-series  models  to  forecast  the  behaviors  of  those
counterparties.

We focus  on modeling the  behaviors of three nonmonetary  counterparties that
can have  an influence  on liquidity conditions.  First, the  Government could
have  an impact  on liquidity  via the  transfers between  its account  at the
central bank and those of the  banks due to public expenditures and collection
of revenues.  Second, the public  influences liquidity by  demanding banknotes
and coins  which are issued  against banks’  reserves. Third, if  the exchange
rate is  fixed and capital transactions  allowed, a broad set  of resident and
non-resident  counterparties   could  possibly  influence  liquidity   via  FX
purchases and  sales from and  to the central bank.  The items in  the central
bank balance  sheet under  the control of  its nonmonetary  counterparties are
called “autonomous liquidity factors.”

Currency in  Circulation follows  distinct seasonal  patterns due  to calendar
effects  related   to  the  work  week   as  well  as  public   and  religious
holidays. These effects  make it possible to develop  point forecasting models
for  the level  of  these factors.  For this  reason,  we propose  forecasting
currency in circulation using models  […]. Forecasting currency in circulation
involved specific challenges such as treating mobile holiday (e.g., Ramadan or
Chinese New  Year) and  structural break  due to external  shocks such  as the
COVID-19 pandemic.

On the other hand, net foreign assets do not display seasonal patterns, but do
exhibit conditional  volatility characteristic of financial  returns data. For
this  reason, we  propose forecasting  net  foreign assets,  using models  for
conditional heteroskedasticity.   Finally, the state treasury  account follows
seasonal patterns  for some of  its determinants but also  exhibit conditional
volatility characteristics for others. Some items that determined the Treasury
account  have clear  seasonal  pattern such  as the  payment  of salaries  and
pensions  as well  as the  collection  of taxes.  On the  other hand,  capital
expenditures and tax revenues usually do not follow regular patterns.

In addition to  developing forecasting models for  three individual autonomous
factors,  we also  consider  forecast combination  through  taking an  equally
weighted  combination  either  of  all  forecasts  or  of  a  trimmed  set  of
models. The forecast quality is tested based on predictive performance metrics
reflecting accuracy,  bias, and reliability,  which will be explained  in this
paper.  Models are,  then, ranked  based  on their  predictive performance  to
combine them.

Finally, we also apply the method  of forecast reconciliation to the liquidity
forecasting  problem.  This  involves,  first, generating  forecasts  for  the
aggregate  (or  net)  liquidity  due   to  net  foreign  assets,  currency  in
circulation and state account balance, leading to four forecasts (one for each
autonomous factor and one of the aggregates). In general, these four forecasts
will not be  coherent, i.e., the aggregate  forecast will not be  equal to the
aggregate of  the three  individual factors.  Reconciliation adjusts  the four
forecasts to ensure coherence. This method has been shown to improve forecasts
in several contexts and is applied to liquidity forecasting for the first time
here.


The remainder of the paper is  summarized as follows. Section 2 introduces the
data  on  the autonomous  factors,  highlighting  the  main features  of  each
factor. Section 3 introduces the models  and methods used both for forecasting
and  forecast reconciliation.  Section  4 presents  the  results of  extensive
forecast evaluation and Section 7 concludes .

%%% Local Variables:
%%% mode: latex
%%% TeX-master: t
%%% End:
